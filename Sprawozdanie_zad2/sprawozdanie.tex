\documentclass{classrep}
\usepackage{color}
\usepackage{url}
\usepackage{hyperref}
\usepackage{amsmath}
\usepackage[T1]{fontenc}
\usepackage{polski}
\usepackage[utf8]{inputenc}
\usepackage{graphicx}
\graphicspath{ {./rys/} }

\usepackage{etoolbox}
\let\bbordermatrix\bordermatrix
\patchcmd{\bbordermatrix}{8.75}{4.75}{}{}
\patchcmd{\bbordermatrix}{\left(}{\left[}{}{}
\patchcmd{\bbordermatrix}{\right)}{\right]}{}{}

\studycycle{Informatyka, studia dzienne, I st.}
\coursesemester{VI}

\coursename{Komputerowe systemy rozpoznawania}
\courseyear{2019/2020}

\courseteacher{dr inż. Marcin Kacprowicz}
\coursegroup{poniedziałek, 12:00}

\author{
	\studentinfo{Radosław Grela}{216769} \and
	\studentinfo{Jakub Wąchała}{216914} 
}

\title{Zadanie 2: Lingwistyczne podsumowania baz danych}


\begin{document}
	\maketitle
	
	\section{Cel} % Cel
	
	
	\section{Wprowadzenie} % Wprowadzenie
	
	\section{Opis implementacji} % Opis implementacji
	Program został stworzony w języku C\#. Graficzny interfejs użytkownika został stworzony przy wykorzystaniu Windows Presentation Foundation. \ldots 
	
	\section{Materiały i metody} % Materiały i metody
	\subsection{Baza danych}
	Do przeprowadzania badań oraz do generowania podsumowań wykorzystaliśmy bazę danych dotyczącą piłkarzy z gry FIFA 20. Pochodzi ona ze źródła \cite{baza}. Składa się ona z 18278 rekordów posiadających 104 atrybuty. Do naszego projektu skorzystamy z 11. Są to następujące atrybuty:
	
	\begin{enumerate}
		\item Wiek - \textsl{age} - wartość z przedziału [16, 42] \\
		Przykładowe zmienne lingwistyczne dla wieku: 
		\begin{itemize}
			\item \textsl{(16-21) bardzo młody}
			\item \textsl{(22-25) młody}
			\item \textsl{(26-32) średni}
			\item \textsl{(33-42) stary}
		\end{itemize}
		Należy zauważyć, że wiek w przypadku zawodnika piłki nożnej oceniany jest w inny sposób niż wiek przeciętnego człowieka.
		\item Wzrost (w cm) - \textsl{height\_cm} - wartość z przedziału [156, 205] \\
		Przykładowe zmienne lingwistyczne dla wzrostu: 
		\begin{itemize}
			\item \textsl{(156-166) niski}
			\item \textsl{(167-177) średni}
			\item \textsl{(178-188) wysoki}
			\item \textsl{(189-205) bardzo wysoki}
		\end{itemize}
		\item Waga (w kg) - \textsl{weight\_kg} - wartość z przedziału [50, 110]
		\item Ocena ogólna - \textsl{overall} - wartość z przedziału [48, 94]
		\item Wartość zawodnika (w EUR) - \textsl{value\_eur} - wartość z przedziału [0, 106 000 000]
		\item Wykończenie - \textsl{attacking\_finishing} - wartość z przedziału [2, 95]
		\item Dribbling - \textsl{skill\_dribbling} - wartość z przedziału [4, 97]
		\item Podkręcenie piłki - \textsl{skill\_curve} - wartość z przedziału [6, 94]
		\item Długie podania - \textsl{skill\_long\_passing} - wartość z przedziału [8, 92]
		\item Sprint - \textsl{movement\_sprint\_speed} - wartość z przedziału [11, 96] \\
		Przykładowe zmienne lingwistyczne dla sprintu: 
		\begin{itemize}
			\item \textsl{(11-30) bardzo wolny}
			\item \textsl{(31-55) wolny}
			\item \textsl{(56-70) średni}
			\item \textsl{(71-85) szybki}
			\item \textsl{(86-96) bardzo szybki}
		\end{itemize}
		\item Siła strzału - \textsl{power\_shot\_power} - wartość z przedziału [14, 95]
	\end{enumerate}

	Każda z kolumn jest typu całkowitego.
	
	
	\section{Wyniki} % Wyniki
	
	\section{Dyskusja} % Dyskusja
	
	\section{Wnioski}
		
	
	\begin{thebibliography} {0}
		\bibitem{anbook} Niewiadomski, Adam. Methods for the Linguistic Summarization of Data: Applications of Fuzzy Sets and Their Extensions. Akademicka Oficyna Wydawnicza EXIT. Warszawa, 2008. ISBN 978-83-60434-40-6
		\bibitem{baza} https://www.kaggle.com/stefanoleone992/fifa-20-complete-player-dataset
		
	\end{thebibliography}
\end{document}
