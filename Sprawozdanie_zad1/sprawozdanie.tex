\documentclass{classrep}
\usepackage{color}
\usepackage{url}
\usepackage{hyperref}
\usepackage[T1]{fontenc}
\usepackage[polish]{babel}
\usepackage[utf8]{inputenc}
\selectlanguage{polish}

\studycycle{Informatyka, studia dzienne, I st.}
\coursesemester{VI}

\coursename{Komputerowe systemy rozpoznawania}
\courseyear{2019/2020}

\courseteacher{dr inż. Marcin Kacprowicz}
\coursegroup{poniedziałek, 12:00}

\author{
  \studentinfo{Radosław Grela}{216769} \and
  \studentinfo{Jakub Wąchała}{216914} 
}

\title{Zadanie 1: ekstrakcja cech, miary podobieństwa, klasyfikacja}
\svnurl{https://github.com/Bonniu/KSR}

\begin{document}
\maketitle

\section{Cel}
Celem naszego zadania było stworzenie aplikacji do klasyfikacji tekstów za pomocą metody k-NN (k najbliższych sąsiadów) oraz
różnych metryk i miar podobieństwa, a następnie porównać kategorie z tymi wygenerowanymi przez aplikację.

\section{Wprowadzenie}
Głównym zagadnieniem projektowym, z którym mieliśmy do czynienia w ramach zadania 1 była klasyfikacja statystyczna tekstów na podstawie wektora wyekstrahowanych cech. Do przeprowadzenie eksperymentu zaimplementowaliśmy algorytm \textsl{k-najbliższych sąsiadów}.

Algorytm k-najbliższych sąsiadów \textsl{(k-NN - k-nearest neighbors)} to jeden z algorytmów zaliczanych do grupy algorytmów leniwych. Jest to taka grupa algorytmów, która szuka rozwiązania dopiero, gdy pojawia się wzorzec testujący. Przechowuje wzorce uczące, a dopiero później wyznacza się odległość wzorca testowego względem wzorców treningowych. \cite{leniwy}. 

Algorytm ten działa w taki sposób, że dla każdego wzorca testowego obliczana jest odległość za pomocą wybranej wetryki względem wzorców treningowych, a następnie wybierana jest k najbliższych wzorców treningowych. Wynik wyznaczony jest jako najczęstszy element wśród nich. W naszym zadaniu odległość ta jest równa skali podobieństwa tekstów. 

\subsection{Ekstrakcja cech}
Do ekstrakcji cech charaklterystycznych tekstu utworzyliśmy wektor cech, który opisuje tekst za pomocą 10 cech. Liczba słów zawsze jest liczona po zastosowaniu stop-listy oraz stemizacji, bez znaków przestankowych.
\begin{itemize}
\item[•] $C_1$ - Stosunek słów kluczowych do wszystkich słów w pierwszych 10\% tekstu. Obliczona jest za pomocą wzoru:
\begin{equation} C_1 = s_{k10} / s_{10}  \end{equation} gdzie $s_{k10}$ to liczba słów kluczowych, a $s_{10}$ to liczba wszystkich słów w pierwszych 10\% tekstu.
\item[•] $C_2$ - Stosunek słów kluczowych do wszystkich słów w ostatnich 10\% tekstu. Obliczona jest za pomocą wzoru:
\begin{equation} C_2 = s_{k90} / s_{90}  \end{equation} gdzie $s_{k90}$ to liczba słów kluczowych, a $s_{90}$ to liczba wszystkich słów w ostatnich 10\% tekstu.
\item[•] $C_3$ - Stosunek słów kluczowych do wszystkich słów w dokumencie. Obliczona jest za pomocą wzoru:
\begin{equation} C_3 = s_k / s  \end{equation} gdzie $s_k$ to liczba słów kluczowych, a $s$ to liczba wszystkich słów w dokumencie.
\item[•] $C_4$ - Stosunek słów kluczowych, których ilość liter $\in$ (0,4] do wszystkich słów w dokumencie. Obliczona jest za pomocą wzoru:
\begin{equation} C_4 = s_k / s  \end{equation} gdzie $s_k$ to liczba słów kluczowych,  których ilość liter $\in$ (0,4], a $s$ to liczba wszystkich słów w dokumencie.
\item[•] $C_5$ - Stosunek słów kluczowych, których ilość liter jest $\geq$8 do wszystkich słów w dokumencie. Obliczona jest za pomocą wzoru:
\begin{equation} C_5 = s_k / s  \end{equation} gdzie $s_k$ to liczba słów kluczowych, a $s$ to liczba wszystkich słów w dokumencie.
\end{itemize}


%Stosunek słów kluczowych do wszystkich słów w pierwszych 10% tekstu
%Stosunek słów kluczowych do wszystkich słów w ostatnich 10% tekstu
%Stosunek słów kluczowych do wszystkich słów w dokumencie
%Stosunek słów kluczowych gdzie ilość liter (0,4] do wszystkich słów
%Stosunek słów kluczowych do wszystkich słów gdzie ilość liter słów kluczowych 8+
%Stosunek linii do ilości akapitów
%Stosunek słów o długości >6 do wszystkich słów
%Stosunek słów o długości <=6 do wszystkich słów
%Ilość słów unikalnych
%Ilość słów, których długość wynosi [5,8]

{\color{blue}
- Z jakiego zakresu/zbioru  cecha przyjmuje wartosci przed normalizacją.
- Czy ,,długość'' oznacza liczbę liter, a może słów? Czy jest obliczana przed czy po stemizacji i/lub zastosowaniu stoplisty? jeśli znaków, to czy znaki przestankowe także są zliczane?
- W przypadku cech mniej intuicyjnych (a najlepiej wszystkich) - mile widziany przykład jak liczyć (może być jeden tekst dla wszystkich cech pod ich opisem).
- Jakie znaczenie ma ta cecha tekstu dla jego rozpoznania? Czy np. im tekst dłuższy, tym bardziej związany z etykietą USA lub CANADA? (istotne!)}

\section{Opis implementacji}
{\color{blue}
Należy tu zamieścić krótki i zwięzły opis zaprojektowanych klas oraz powiązań
między nimi. Powinien się tu również znaleźć diagram UML  (diagram klas)
prezentujący najistotniejsze elementy stworzonej aplikacji. Należy także
podać, w jakim języku programowania została stworzona aplikacja. }

\section{Materiały i metody}
{\color{blue}
W tym miejscu należy opisać, jak przeprowadzone zostały wszystkie badania,
których wyniki i dyskusja zamieszczane są w dalszych sekcjach. Opis ten
powinien być na tyle dokładny, aby osoba czytająca go potrafiła wszystkie
przeprowadzone badania samodzielnie powtórzyć w celu zweryfikowania ich
poprawności (a zatem m.in. należy zamieścić tu opis architektury sieci,
wartości współczynników użytych w kolejnych eksperymentach, sposób
inicjalizacji wag, metodę uczenia itp. oraz informacje o danych, na których
prowadzone były badania). Przy opisie należy odwoływać się i stosować do
opisanych w sekcji drugiej wzorów i oznaczeń, a także w jasny sposób opisać
cel konkretnego testu. Najlepiej byłoby wyraźnie wyszczególnić (ponumerować)
poszczególne eksperymenty tak, aby łatwo było się do nich odwoływać dalej.}

\section{Wyniki}
{\color{blue}
W tej sekcji należy zaprezentować, dla każdego przeprowadzonego eksperymentu,
kompletny zestaw wyników w postaci tabel, wykresów itp. Powinny być one tak
ponazywane, aby było wiadomo, do czego się odnoszą. Wszystkie tabele i wykresy
należy oczywiście opisać (opisać co jest na osiach, w kolumnach itd.) stosując
się do przyjętych wcześniej oznaczeń. Nie należy tu komentować i interpretować
wyników, gdyż miejsce na to jest w kolejnej sekcji. Tu również dobrze jest
wprowadzić oznaczenia (tabel, wykresów) aby móc się do nich odwoływać
poniżej.}

\section{Dyskusja}
{\color{blue}
Sekcja ta powinna zawierać dokładną interpretację uzyskanych wyników
eksperymentów wraz ze szczegółowymi wnioskami z nich płynącymi. Najcenniejsze
są, rzecz jasna, wnioski o charakterze uniwersalnym, które mogą być istotne
przy innych, podobnych zadaniach. Należy również omówić i wyjaśnić wszystkie
napotakane problemy (jeśli takie były). Każdy wniosek powinien mieć poparcie
we wcześniej przeprowadzonych eksperymentach (odwołania do konkretnych
wyników). Jest to jedna z najważniejszych sekcji tego sprawozdania, gdyż
prezentuje poziom zrozumienia badanego problemu.}
\section{Wnioski}
{\color{blue}W tej, przedostatniej, sekcji należy zamieścić podsumowanie
najważniejszych wniosków z sekcji poprzedniej. Najlepiej jest je po prostu
wypunktować. Znów, tak jak poprzednio, najistotniejsze są wnioski o
charakterze uniwersalnym.}


\begin{thebibliography} {0}
\bibitem{leniwy} \textsl{http://home.agh.edu.pl/~horzyk/lectures/miw/KNN.pdf} [dostęp 17.03.2020]
\end{thebibliography}
\end{document}
